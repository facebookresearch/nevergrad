

The main unexpected result is the excellent performance of quasi-opposite sampling\cite{quasiopposite} (see QODE, QNDE,
QOPSO, SQOPSO in Section \ref{bigstats}). We adapted it
from DE to Particle Swarm Optimization, with SQOPSO using, for each particle $p$ with speed $v$, another particle
$-r\times p$ and speed $-r\times v$. We believe that this is particularly true for benchmarks which do not have a single
scale for optima. If we define optima by e.g. a multivariate normal distribution with mean zero and identity covariance matrix, then
in large dimension the optimum always has a norm close to $\sqrt{dimension}$: this is not what happens in many
real-world benchmarks. Similar phenomenons when all coordinates have the same distribution and are independent imply
that benchmarks sometimes do not evaluate the robustness to scaling. Quasi-opposite sampling might be the tool for
ensuring robustness in continuous optimization, in front of scaling uncertainties.

As in SAT competitions, we observe excellent results for wizards, such as NGOpt. All methods performing well on a wide
range of benchmarks are wizards. Note that the best results obtained at the BBO Challenge\cite{bbochallenge}, which had
the best performance for a method combining DE, SMAC and others.

SMAC got better results than other Bayesian Optimization methods. Bayesian Optimization methods are limited to low
budget / dimension contexts, and a strong competitor for continuous optimization with low budget is Cobyla.

Results are visible in the discrete case (robust good performance for \cite{relengler,lengler}) and in multi-objective
optimization (good performance overall for \cite{pde,mode}), but further work is needed on this part. Noisy optimization
is tricky, and rules for choosing between completely different methods such as sequential quadratic programming or
TBPSA\cite{vasilfoga} and others might be needed.




%Unless specified otherwise, MetaModel means CMA equipped with a quadratic MetaModel, which is automatically enabled when the learning performs well.
%Neural, RF and SVM as prefixes for MetaModel mean that we use Neural nets, Random Forest or Support Vector Machines as
%surrogate models. The quadratic and random forest meta-models look best overall. DE or OnePlusOne as a suffix mean that
%we use differential evolution or the $(1+1)$ evolution strategy as an underlying optimization model, instead of CMA.
%ChainMetaModelSQP is a memetic algorithm: it combines MetaModel (hence, CMA plus a MetaModel) and a final run using a local method, namely SQP (sequential quadratic programming).
%Usually, a meta-model and a fast local search a the end do improve evolutionary methods in continuous domains. 
%
%On benchmarks close to the good old BBOB, CMA and variants do perform well. The variants equipped with a MetaModel perform better, and
%variants equipped with the the MetaModel and the final local search (namely ChainMetaModelSQP) are even better on YABBOB. It
%remains the best with big budget (YABIGBBOB), In the parallel case, many tools are nearly equivalent, and in the small
%budget case (YASMALLBBOB, YATINYBBOB, YATUNINGBBOB) Cobyla performs well, as frequently in the state of the art.%\cite{TODO}.
%
%On many real-world benchmarks, the budget is lower than in the traditional context of BBOB with budget $=$ dimension $\times$ 1000. There
%are cases with a ratio budget/dimension $<1$, of the order of a few units or a few dozens. DE performs well in many
%cases. This is consistent with many publications considering real-world problems. %TODO
%However for the minimum ratio budget/dimension the simple $(1+1)$ evolution strategy, in continuous domains, is
%excellent.
%We note interesting performances of QODE (quasi-opposite DE) in many cases.
%
%SQP is excellent in noisy optimization, but methods with ad hoc scale can compete: the scale is critical in noisy
%optimization, unless the budget is large. In some cases, TBPSA (combining population-control\cite{mlis} and other tools
%as in \cite{vasilfoga}) also performs well, as well as combinations between bandits and evolutionary algorithms
%(NoisyDiscreteOnePlusOne)..
%Methods designed in the noisy case might diverge.
%
%We include benchmarks with one or several or many constraints (prefix onepen, pen and megapen), tackled with dynamic
%penalization: results were not fundamentally different from the non-penalized case. However, MetaModels are effective in
%a very stable and visible manner: this is consistent with the state of the art.% \cite{TODO}. 
%Cobyla remains very strong in low budget cases. Bayesian Optimization methods are more expensive than other algorithms,
%but rarely perform well, though they are comparable to other methods in low budget cases. The different Bayesian
%Optimization methods perform differently from each other, which is consistent with their highly parametric nature.
%
%%MetaModel perform well on BBOB-style optimization, but were also excellent for several low budget things.
%
%In discrete optimization, INSTRUM\_DISCRETE and SEQUENTIAL\_INSTRUM\_DISCRETE are benchmarks with arity $> 2$ and in
%which the order can not be exploited. Methods which ignore the order do perform better.
%We also consider PBO and BONNANS test functions. Regarding discrete contexts, we note the great performance of the so-called ``DiscreteLenglerOnePlusOne''
%\cite{relengler,lengler}. We tested variants with a different constant (methods with ``Lengler'' and one of ``Fourth, Half, 2, 3''
%in the name) and it turns out that the proved constant, in spite of the simple context in which it was derived, is good.
%Adding ``Tabu'' (SA, as self-avoiding, in the name) can make methods better.
%Still in the discrete case, we note the good performance of methods with ``Recombining'' in the name: while it was less
%investigated theoretically than variants of the discrete $(1+1)$ method, methods with crossover might be quite
%competitive.
%
%Note that Cobyla is frequently at the top, as well as ChainMetaModelSQP, OnePlusOne, DE, RFMetaModel, or DiscreteLenglerOnePlusOne.
%It looks likely that wizards could be improved by using more of these, in particular the random forest metamodel which
%is rarely used.
%Overall, wizards (such as NGOpt or NGOptRW) do perform well, though they are not perfect. Independent tests on other
%benchmarks will be interesting as the wizards tested here have been alive in the platform for quite a long time. They
%are useful, but can guess wrong sometimes, so human expertise is still quite useful.
%
%Regarding the principles of benchmarking, we note that the two different views in Nevergrad (the heatmap and the average
%normalized loss) present different views, with sometimes significant differences. This emphasizes how much how we look at data has a big impact on the interpretation.
%
%%Regarding BBOB variants, TODO
%%
%%Regarding PBBOB, TODO
%%
%%Regarding Holland and Voronoi crossovers, TODO
%%
%%Algorithms taking into account the difference between ordered and unordered discrete variables TODO
%%
%%In the continuous low budget case, Cobyla and BOBYQA are frequently excellent in moderate dimension. The high-dimensional case sees great successes of DE variants.
%
%Consistently with some real world experiments in \cite{micropredictions1,micropredictions2}, we note the the $(1+1)$ evolution strategy with one-fifth rule from \cite{rechenberg73} is still quite good. In artificial benchmarks with a lot of evaluations, high conditionning and rotated contexts, it can become weak: for many realistic contexts, in particular a realistic ratio budget/dimension, it is quite good.
%The simple $(1+1)$ evolution strategy with one-fifth rule, even more when equipped with a meta-model, turned out to be powerful.
%
%In real-world benchmark tunings, results varied vastly from one benchmark to the next.  It looks like we can conclude 
%anything we want by selecting a benchmark or by tuning algorithms for a specific case. Overall, maybe DE variants and
%BOBYQA, HyperOpt, RandomSearch perform well.
%
%The multi-objective setting is quite difficult to analyze. The DE adapted to the multi-objective case available in
%Nevergrad\cite{pde,mode} performs well in classical settings, but high dimensional and many objective cases lead to surprisingly good
%results  by e.g. discrete methods running in the continuous case (which means they are handled that variables with
%infinitely many possibles values). This deserves additional investigation, as using discrete algorithms such as
%DiscreteOnePlusOne in continuous difficult problems is rather new and ignores the topology of each variable as if the
%real numbers were not ordered: this means randomly redrawing some variables.
%
%BFGS can be used in a black-box setting, using finite differences. This is, however, rarely a good option.
%
%Some benchmarks such as photonics or topology optimization have variables organized in arrays of dimension 1 or 2. Not
%all algorithms are able to use this.
%
%GOMEA was recently added, with various linkage options. Results are good e.g. for high-dimensional low-budget problems.
%
%\subsection{Caveats, further work}
%Some benchmarks were implemented but not included in the release due to legal issues in the license. 
%We did not include the important case in which a multi-objective run is performed on surrogate models only: (1) randomly sample, (2) approximate the objective functions by surrogate models, (3) perform a multi-objective optimization on the surrogate only. This is useful for including the user in the loop. This is not tested in the current benchmarks.
%
%Compared to the old Dashboard from 2021, results are somehow similar, with more details. However, we have more real world benchmarks and more discrete experiments. Also,
%some methods have been removed, in particular some slow methods which were rarely performing well compared to present methods.
%
%Key points in benchmarks are frequently how the initialization matches the distribution of the optima, and various
%benchmarks are too close to a single algorithm: this can be the case in the present work as well, in particular for
%wizards. Therefore, independent runs with your preferred modifications of the settings will be fruitful for us: we made our best
%for making this user-friendly.
